\documentclass[10pt,twocolumn,letterpaper]{article}
\usepackage[utf8]{inputenc}
\usepackage[english]{babel}
\usepackage{listings}
\lstset{
  language=bash,
  basicstyle=\ttfamily
}
%lstinputlisting
\usepackage{hyperref}
\setlength\parindent{0pt}
\begin{document}
\title{Setup Jaeger on Centos in MOC}
\author{Bowen Song\thanks{sbowen@bu.edu}}
\maketitle

\tableofcontents

\clearpage

\section{Introduction}
This document is to illustrate an example of how to setup Jaeger on Centos in MOC. The setup involves 2 VMs, VM6 and VM7. Setup would involve Jaeger and Docker on both VMs. \\

To demonstrate an understanding, I am deploying HOTRod on the VMs. Front end of HOTRod would be on VM6 and All other components would be on VM7.\\

This would demonstrate my understating of HOTRod and Jaeger in the attempt to trace HOTRod application. \\
\section{SSH Into VMswith respect to current configuration}
`ssh centos@128.31.25.229 -A' ssh into VM1\\
`ssh centos@192.168.100.5 -A' ssh into VM6\\
`ssh centos@192.168.100.6 -A' ssh into VM7\\
Password: 1234 for VM 6 and VM 7\\
\section{Install Docker on Centos}
Followed Document for installing Docker on Centos from:
\url{https://docs.docker.com/install/linux/docker-ce/centos/#os-requirements}.\\

`sudo dockerd' to start Docker daemon\\
`sudo' attempt the following command to run docker

\url{https://jaeger.readthedocs.io/en/latest/getting_started/}\\

\section{Show Docker Jaeger GUI}
The GUI is shwon on \url{http://localhost:16686}. To show this GUI, use X11 with Firefox from VM1 to VM6/7.
Steps here can be mirrored in VM6.
\begin{enumerate}
	\item Setup X11 on local Computer
	% \item Specified `X11Forwarding yes'  in \path{/etc/ssh/sshd_config} in VM 7
	% \item Specified `ForwardAgent  yes'  in \path{/etc/ssh/sshd_config} in VM 7
	% \item Specified `ForwardX11Trusted yes'  in \path{/etc/ssh/sshd_config} in VM 7
	\item Changes to \url{/etc/ssh/sshd_config} in VM7 should be: \begin{lstlisting}
AllowAgentForwarding yes
X11Forwarding yes
	\end{lstlisting}
	\item ``sudo yum install xorg-x11-xauth xterm'' install X11 on server VM 1
	\item ``systemctl enable sshd.service'' to allow sshd.service
	\item ``systemctl restart sshd.service'' restart sshd.service for ssh\_config to work
	\item ``systemctl status sshd.service''  check status if you want
	\item ``ssh -X centos@192.168.100.6 -A'' into VM 7 and run firefox to see GUI (``sudo yum install firfox'' if needed)
	\item ``sudo yum groupinstall GNOME Desktop'' to allow display 
	\item ``sudo yum groupinstall 'X Window System' ''to sitnall X11
	\item ``sudo yum install xorg-x11-xauth'' on VM7 for x11 authentication
	\item ``sudo yum search xorg-x11''
	\item In VM 7 sudo vim .bashrc and add ``export DISPLAY=localhost:10.0''
	\item The .bash\_profile  should look like \begin{lstlisting}
# .bash_profile

# Get the aliases and functions
if [ -f ~/.bashrc ]; then
        . ~/.bashrc
fi

PATH=$PATH:$HOME/.local/bin:$HOME/bin:
export PATH
export GOBIN="$HOME/projects/bin"
export GOPATH="$HOME/projects/src"
	\end{lstlisting}

\end{enumerate}

\section{Deploy HOTRod on Centos}
As a key element, install \textit{GO} \textbf{1.10 or newest} on VMs,
follow the instrucations on \url{https://www.digitalocean.com/community/tutorials/how-to-install-go-1-7-on-centos-7}.\\

Install \textit{shasum} for checking if GO is downloaded correctly. (Skipped in this case)\\

Now get HOTROD
\begin{enumerate}
\item install git ``sudo yum install git''
\item Follow the \url{https://github.com/jaegertracing/jaeger/tree/master/examples/hotrod}

\item for installing go dep, ``go build -o \$GOPATH/src/github.com/TrueFurby/go-callvis/"
\end{enumerate}


\end{document}

